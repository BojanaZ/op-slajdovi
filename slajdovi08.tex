\documentclass[utf8,compress,aspectratio=169]{beamer}
\usepackage{irbookslide}
\usepackage{irilmenau2}
\usepackage{url}
\usepackage{fontspec} % zahteva paket euenc
\usepackage{xunicode}
\usepackage{xltxtra}
\usepackage{polyglossia}
\usepackage[cache=false]{minted}
\usepackage{xcolor,colortbl}
\usepackage{textcomp}
\usepackage{unicode-math}

\title{Kolekcije podataka}
\subtitle{\tiny{Slajdovi za predmet Osnove programiranja}}
\subject{Osnove programiranja}
\institute{Katedra za informatiku, Fakultet tehničkih nauka, Novi Sad}
\date{2021.}

\begin{document}

\frame{\titlepage}

\frame{
  \frametitle{Ciljevi}
  \begin{itemize}
    \item upotreba lista (nizova) za reprezentaciju kolekcije povezanih podataka
    \item poznavanje funkcija za manipulaciju Python listama
    \item upotreba lista za upravljanje podacima
    \item upotreba lista i rečnika za organizaciju složenih podataka
  \end{itemize}
}

\section[Primer]{Primer: statistika}

\frame{
  \frametitle{Primer: jednostavna statistika}
  \begin{itemize}
    \item mnogi programi rukuju velikom količinom sličnih podataka
    \begin{itemize}
      \item reči u tekstu
      \item spisak studenata
      \item podaci iz eksperimenta
      \item poslovni partneri
      \item grafički objekti na ekranu
      \item karte u špilu
    \end{itemize}
  \end{itemize}
}

\begin{frame}[fragile,shrink]
  \frametitle{Primer od ranije}
  \begin{itemize}
    \item izračunavanje proseka za niz brojeva, korišćenjem praznog stringa kao sentinela
  \end{itemize}
\begin{minted}{python}
def main():
    sum = 0.0
    count = 0
    xStr = input("Unesite broj (<Enter> za kraj) >> ")
    while xStr != "":
        x = eval(xStr)
        sum = sum + x
        count = count + 1
        xStr = input("Unesite broj (<Enter> za kraj) >> ")
    print("\nProsek je", sum / count)
\end{minted}
\end{frame}

\begin{frame}[fragile]
  \frametitle{Primer od ranije $_2$}
  \begin{itemize}
    \item unosimo niz brojeva, ali program ne pamti \myblue{sve} brojeve koje smo uneli
    \item pamti samo tekuću sumu unetih brojeva
    \item sada je potrebno izračunati prosek, \myblue{median} i \myblue{standardno odstupanje}
  \end{itemize}
\end{frame}

\begin{frame}[fragile]
  \frametitle{Median}
  \begin{itemize}
    \item \myblue{median} je vrednost koja deli skup brojeva na dva jednako velika dela
    \item za brojeve 2, 4, 6, 9, 13 median je 6
    \begin{itemize}
      \item ima dve vrednosti manje od 6 i dve veće od 6
      \item broj 6 ,,deli`` ovaj niz na dva jednaka dela
      \item broj 6 nalazi se u sredini sortiranog niza brojeva
    \end{itemize}
    \item izračunavanje mediana:
    \begin{itemize}
      \item unesi sve brojeve
      \item sortiraj po veličini
      \item izdvoj srednji broj (ne srednju vrednost!)
    \end{itemize}
  \end{itemize}
\end{frame}

\begin{frame}[fragile]
  \frametitle{Standardno odstupanje}
  \begin{itemize}
    \item \myblue{standardno odstupanje} je mera koliko podaci odstupaju od prosečne vrednosti
    \item ako su svi podaci ,,blizu`` prosečne vrednosti, odstupanje je malo
    \item ako su podaci ,,rasuti``, odstupanje je veliko
    \item izraz za standardno odstupanje: \\
    $$ s = \sqrt{\frac{\sum{(\bar{x}-x_i)^2}}{n-1}} $$
    \item $\bar{x}$ je prosek
    \item $n$ je broj podataka
    \item $x_i$ je $i$-ti podatak
    \item izraz $(\bar{x} - x_i)^2$ je kvadrat ,,odstupanja`` pojedinačnog podatka od proseka
  \end{itemize}
\end{frame}

\begin{frame}[fragile]
  \frametitle{Standardno odstupanje $_2$}
  \begin{itemize}
    \item za niz brojeva 2, 4, 6, 9, 13
    \item prosek je 6.8 \\
    $$(6.8-2)^2 + (6.8-4)^2 + (6.8-6)^2 + (6.8-9)^2 + (6.8-13)^2 = 149.6$$
    $$s = \sqrt{\frac{149.6}{5-1}} = \sqrt{37.4} = 6.11$$
  \end{itemize}
\end{frame}

\begin{frame}[fragile]
  \frametitle{Standardno odstupanje $_3$}
  \begin{itemize}
    \item za izračunavanje standardnog odstupanja potreban je prosek
    \item (koji ne možemo izračunati dok nemamo sve podatke)
    \item ali i svaki pojedinačni podatak! \\ \ \\
    \item treba nam način da zapamtimo sve podatke koji se unose
  \end{itemize}
\end{frame}

\section[Lista]{Lista}

\begin{frame}[fragile]
  \frametitle{Lista}
  \begin{itemize}
    \item treba nam rešenje za skladištenje čitave kolekcije brojeva
    \item ne možemo da koristimo posebne promenljive jer ne znamo unapred koliko će brojeva biti
    \item treba nam način da \myblue{ceo niz} brojeva tretiramo kao jednu promenljivu
  \end{itemize}
\end{frame}

\begin{frame}[fragile]
  \frametitle{Python lista}
  \begin{itemize}
    \item Python \myblue{lista} je uređena sekvenca podataka
    \item na primer, sekvenca $n$ brojeva može se nazvati S: \\
    $$S = s_0, s_1, s_2, \ldots, s_{n-1}$$
    \item pojedine vrednosti u listi možemo dobiti pomoću \myblue{indeksa}
    \item korišćenjem indeksa, možemo sažeto zapisati matematičke operacije nad svim elementima liste
    \item na primer, zbir svih elemenata liste: \\
    $$ \sum_{i=0}^{n-1}{s_i} $$
  \end{itemize}
\end{frame}

\begin{frame}[fragile]
  \frametitle{Python lista $_2$}
  \begin{itemize}
    \item neka je sekvenca smeštena u promenljivoj \texttt{s}
    \item petlja za sumiranje svih elemenata liste:
  \end{itemize}
\begin{minted}{python}
sum = 0
for i in range(n):
    sum = sum + s[i]
\end{minted}
  \begin{itemize}
    \item skoro svi programski jezici poznaju ovakvu strukturu
    \item negde se zove \myblue{niz}
  \end{itemize}
\end{frame}

\begin{frame}[fragile]
  \frametitle{Python lista $_3$}
  \begin{itemize}
    \item \myblue{lista} (niz) je sekvenca vrednosti gde cela sekvenca ima svoj identifikator (npr. \texttt{s})
    \item gde se elementima sekvence pristupa pomoću indeksa \\(npr. \texttt{s[i]}) \\ \ \\
    \item u drugim programskim jezicima nizovi su obično \myblue{fiksne dužine} \\i \myblue{homogeni}
    \begin{itemize}
      \item kada se niz inicijalizuje, mora se navesti max broj elemenata
      \item svi elementi niza su istog tipa
    \end{itemize}
  \end{itemize}
\end{frame}

\begin{frame}[fragile]
  \frametitle{Python lista $_4$}
  \begin{itemize}
    \item liste u Pythonu nemaju unapred određen broj elemenata
    \begin{itemize}
      \item mogu da se ,,produžavaju`` i ,,skraćuju`` po potrebi \\ \ \\
    \end{itemize}
    \item liste u Pythonu su heterogene
    \begin{itemize}
      \item jedna lista može sadržati elemente različitog tipa \\ \ \\
    \end{itemize}
    \item sadržaj liste se može menjati (\myblue{mutable})
  \end{itemize}
\end{frame}

\begin{frame}[fragile]
  \frametitle{Operacije nad listama}
\begin{center}
\begin{tabular}{l|l}
\textbf{operator} & \textbf{značenje} \\ \hline
\texttt{<list> + <list>} & konkatenacija \\
\texttt{<list> * <int-expr>} & ponavljanje \\
\texttt{<list>[]} & indeksiranje \\
\texttt{len(<list>)} & dužina liste \\
\texttt{<list>[:]} & isecanje \\
\texttt{for <var> in <list>:} & iteracija \\
\texttt{<expr> in <list>} & postojanje elementa
\end{tabular}
\end{center}
\end{frame}

\begin{frame}[fragile]
  \frametitle{Operacije nad listama $_2$}
  \begin{itemize}
    \item osim poslednje operacije, sve ostale su dostupne i za stringove
    \item operacija koja testira postojanje elementa:
  \end{itemize}
\begin{minted}{python}
>>> lst = [1,2,3,4]
>>> 3 in lst
True
\end{minted}
  \begin{itemize}
    \item za razliku od stringova, liste se mogu menjati
  \end{itemize}
\begin{minted}{python}
>>> lst = [1,2,3,4]
>>> lst[2] = 7
>>> lst
[1, 2, 7, 4]
\end{minted}
\end{frame}

\begin{frame}[fragile]
  \frametitle{Operacije nad listama $_3$}
  \begin{itemize}
    \item sumiranje elemenata liste možemo ovako zapisati:
  \end{itemize}
\begin{minted}{python}
sum = 0
for x in s:
    sum = sum + x
\end{minted}
  \begin{itemize}
    \item lista sa jednakim elementima:
  \end{itemize}
\begin{verbatim}
>>> zeroes = [0] * 50
\end{verbatim}
\end{frame}

\begin{frame}[fragile]
  \frametitle{Operacije nad listama $_4$}
  \begin{itemize}
    \item liste se često popunjavaju jedan-po-jedan element
    \item funkcija \texttt{append} dodaje element na kraj liste
  \end{itemize}
\begin{minted}{python}
nums = []
x = eval(input('Unesite broj: '))
while x >= 0:
    nums.append(x)
    x = eval(input('Unesite broj: '))
\end{minted}
  \begin{itemize}
    \item ovde se \texttt{nums} koristi kao akumulator!
    \item inicijalizuje se kao prazna lista
    \item u svakom ciklusu petlje dodaje se jedan element liste
  \end{itemize}
\end{frame}

\begin{frame}[fragile,shrink=5]
  \frametitle{Još operacija nad listama}
\begin{center}
\begin{tabular}{l|l}
\textbf{funkcija} & \textbf{značenje} \\ \hline
\texttt{<list>.append(x)} & dodaj element \texttt{x} na kraj liste \\
\texttt{<list>.sort()} & sortiraj listu \\
\texttt{<list>.reverse()} & obrni redosled elemenata u listi \\
\texttt{<list>.index(x)} & indeks prve pojave vrednosti \texttt{x} u listi \\
\texttt{<list>.insert(i, x)} & ubaci \texttt{x} u listu na mesto indeksa \texttt{i} \\
\texttt{<list>.count(x)} & broj pojavljivanja vrednosti \texttt{x} u listi \\
\texttt{<list>.remove(x)} & ukloni prvu pojavu vrednosti \texttt{x} iz liste \\
\texttt{<list>.pop(i)} & izbaci i-ti element iz liste i vrati njegovu vrednost
\end{tabular}
\end{center}
\end{frame}

\begin{frame}[fragile,shrink=5]
  \frametitle{Još operacija nad listama $_2$}
\begin{minted}{python}
>>> lst = [3, 1, 4, 1, 5, 9]
>>> lst.append(2)
>>> lst
[3, 1, 4, 1, 5, 9, 2]
>>> lst.sort()
>>> lst
[1, 1, 2, 3, 4, 5, 9]
>>> lst.reverse()
>>> lst
[9, 5, 4, 3, 2, 1, 1]
>>> lst.index(4)
2
>>> lst.insert(4, "Hello")
>>> lst
[9, 5, 4, 3, 'Hello', 2, 1, 1]
>>> lst.count(1)
2
>>> lst.remove(1)
>>> lst
[9, 5, 4, 3, 'Hello', 2, 1]
>>> lst.pop(2)
4
>>> lst
[9, 5, 3, 'Hello', 2, 1]
\end{minted}
\end{frame}

\begin{frame}[fragile]
  \frametitle{Još operacija nad listama $_3$}
  \begin{itemize}
    \item većina ovih funkcija ne vraća ništa, već menja sadržaj liste
    \item lista se može produžavati dodavanjem novih vrednosti
    \item ili skraćivati izbacivanjem postojećih
    \item elementi ili čitavi isečci se mogu izbaciti pomoću operatora \texttt{del}
  \end{itemize}
\begin{minted}{python}
>>> myList=[34, 26, 0, 10]
>>> del myList[1]
>>> myList
[34, 0, 10]
>>> del myList[1:3]
>>> myList
[34]
\end{minted}
\end{frame}

\begin{frame}[fragile]
  \frametitle{Principi rada sa Python listama}
  \begin{itemize}
    \item lista je sekvenca podataka koja se čuva kao jedan objekat
    \item elementi liste su dostupni putem indeksa
    \item podliste su dostupne putem isecanja
    \item liste se mogu menjati
    \begin{itemize}
      \item elementi ili isečci se mogu zameniti pomoću dodele vrednosti
      \item nije moguće kod stringova!
    \end{itemize}
    \item liste se mogu produžavati i skraćivati po potrebi
    \item postoje gotove funkcije za manipulaciju listama
  \end{itemize}
\end{frame}

\section[Primer]{Primer: statistika}

\begin{frame}[fragile]
  \frametitle{Statistika sa listama}
  \begin{itemize}
    \item rešićemo naš statistički problem tako što ćemo čuvati sve podatke u listi
    \item napisaćemo funkcije koje računaju prosek, odstupanje i median za datu listu
    \item rešavamo problem korak-po-korak
  \end{itemize}
\end{frame}

\begin{frame}[fragile]
  \frametitle{Funkcija za unos podataka: \texttt{getNumbers}}
  \begin{itemize}
    \item unos brojeva pomoću sentinel petlje
    \item inicijalno prazna lista će biti akumulator
    \item vraćamo listu kao rezultat funkcije kada se unesu svi podaci
    \item upotreba funkcije: \\
    \texttt{data = getNumbers()}
  \end{itemize}
\end{frame}

\begin{frame}[fragile]
  \frametitle{Funkcija za unos podataka: \texttt{getNumbers} $_2$}
\begin{minted}{python}
def getNumbers():
    nums = []     # počnemo sa praznom listom

    # sentinel petlja za unos brojeva
    xStr = input("Unesi broj (<Enter> za kraj) >> ")
    while xStr != "":
        x = eval(xStr)
        nums.append(x)   # dodaj vrednost na kraj liste
        xStr = input("Unesi broj (<Enter> za kraj) >> ")
    return nums
\end{minted}
\end{frame}

\begin{frame}[fragile]
  \frametitle{Funkcija za izračunavanje proseka: \texttt{mean}}
  \begin{itemize}
    \item ulaz: lista brojeva
    \item izlaz: prosečna vrednost elemenata liste
  \end{itemize}
\begin{minted}{python}
def mean(nums):
    sum = 0.0
    for num in nums:
        sum = sum + num
    return sum / len(nums)
\end{minted}
\end{frame}

\begin{frame}[fragile]
  \frametitle{Funkcija za izračunavanje odstupanja: \texttt{stdDev}}
  \begin{itemize}
    \item da bismo izračunali standardno odstupanje, treba nam prosek
    \item da li da izračunavamo prosek unutar \texttt{stdDev}?
    \item da li da vrednost proseka prenesemo kao parametar u \texttt{stdDev}?
  \end{itemize}
\end{frame}

\begin{frame}[fragile]
  \frametitle{Funkcija za izračunavanje odstupanja: \texttt{stdDev} $_2$}
  \begin{itemize}
    \item izračunavanje proseka više puta unutar \texttt{stdDev} nije efikasno
    \item pošto naš program ispisuje i prosek i odstupanje,
    \item prenećemo prosek kao parametar u \texttt{stdDev}
  \end{itemize}
\begin{minted}{python}
def stdDev(nums, avg):
    sumDevSq = 0.0      # akumulator
    for num in nums:
        dev = avg - num
        sumDevSq = sumDevSq + dev * dev
    return sqrt(sumDevSq/(len(nums)-1))
\end{minted}
\end{frame}

\begin{frame}[fragile]
  \frametitle{Funkcija za izračunavanje mediana: \texttt{median}}
  \begin{itemize}
    \item nemamo formulu za računanje mediana: treba nam algoritam
    \item[1] sortiramo brojeve u rastućem redosledu
    \item[2] srednji element niza je median
    \item[3] ako niz ima paran broj elemenata, median je prosek dva srednja elementa
  \end{itemize}
\begin{verbatim}
sortiraj brojeve u rastućem redosledu
if neparan broj elemenata:
    median = srednji element
else:
    median = prosek dva srednja elementa
return median
\end{verbatim}
\end{frame}

\begin{frame}[fragile]
  \frametitle{Funkcija za izračunavanje mediana: \texttt{median} $_2$}
\begin{minted}{python}
def median(nums):
    nums.sort()
    size = len(nums)
    midPos = size // 2
    if size % 2 == 0:
        median = (nums[midPos] + nums[midPos-1]) / 2
    else:
        median = nums[midPos]
    return median
\end{minted}
\end{frame}

\begin{frame}[fragile]
  \frametitle{Glavni program}
\begin{minted}{python}
def main():
    print("Izračunava prosek, median i std. odstupanje.")

    data = getNumbers()
    avg = mean(data)
    std = stdDev(data, avg)
    med = median(data)

    print("\nProsek je", avg)
    print("Standardno odstupanje je", std)
    print("Median je", med)
\end{minted}
\end{frame}

\section[Sekvenca]{Sekvenca}

\begin{frame}[fragile]
  \frametitle{Sekvenca}
  \begin{itemize}
    \item sekvenca je \myblue{immutable} lista
    \item ne možemo joj menjati sadržaj
    \item umesto uglastih zagrada \texttt{[]} koristimo obične \texttt{()}
  \end{itemize}
\begin{minted}{python}
>>> lst = [1,2,3]
>>> lst[0] = 7
>>> lst
[7, 2, 3]
>>> sek = (1,2,3)
>>> sek[0] = 7
Traceback (most recent call last):
  File "<stdin>", line 1, in <module>
TypeError: 'tuple' object does not support item assignment
\end{minted}
\end{frame}

\begin{frame}[fragile]
  \frametitle{Sekvenca $_2$}
  \begin{itemize}
    \item Python efikasnije koristi sekvence od listi
    \item treba koristiti sekvence kad je to moguće
    \item string se ponaša kao sekvenca
  \end{itemize}
\end{frame}


\section[Rečnik]{Rečnik}

\begin{frame}[fragile]
  \frametitle{Rečnik}
  \begin{itemize}
    \item \myblue{rečnik} (dictionary), za razliku od liste, ne poznaje redosled elemenata
    \item elementi rečnika nisu dostupni putem indeksa, već putem \myblue{ključa}
    \item rečnik čuva \myblue{parove ključ-vrednost}
  \end{itemize}
\begin{minted}{python}
>>> lozinke = {"guido":"superprogrammer",
        "turing":"genius", "bill":"monopoly"}
>>> lozinke["guido"]
'superprogrammer'
>>> lozinke["bill"]
'monopoly'
\end{minted}
  \begin{itemize}
    \item šta je ovde ključ, a šta vrednost?
  \end{itemize}
\end{frame}

\begin{frame}[fragile]
  \frametitle{Rečnik $_2$}
  \begin{itemize}
    \item rečnici se mogu menjati
  \end{itemize}
\begin{minted}{python}
>>> lozinke["bill"] = "bluescreen"
>>> lozinke
{'turing': 'genius', 'bill': 'bluescreeen', 'guido': \
'superprogrammer'}
\end{minted}
  \begin{itemize}
    \item originalni redosled nije očuvan!
  \end{itemize}
\end{frame}

\begin{frame}[fragile]
  \frametitle{Rečnik $_3$}
  \begin{itemize}
    \item ključ može biti bilo koji \myblue{immutable} tip - broj, string
    \item vrednost može biti bilo kog tipa \\ \ \\
    \item rečnici se retko javljaju u drugim programskim jezicima
    \item u Pythonu se često koriste
    \item efikasni su, mogu primiti stotine hiljada elemenata
  \end{itemize}
\end{frame}

\begin{frame}[fragile]
  \frametitle{Dodavanje elemenata u rečnik}
  \begin{itemize}
    \item rečnik može biti prazan: \texttt{\{\}}
    \item dodavanje elementa u rečnik: upotrebi se novi ključ
  \end{itemize}
\begin{minted}{python}
>>> lozinke["newuser"] = "jasamnovi"
>>> lozinke
{'turing': 'genius', 'bill': 'bluescreeen', 'guido': \
'superprogrammer', 'newuser': 'jasamnovi'}
\end{minted}
\end{frame}

\begin{frame}[fragile]
  \frametitle{Primer: učitavanje rečnika iz fajla}
\begin{minted}{python}
passwd = {}
for line in open('passwords','r').readlines():
    user, pass = string.split(line)
    passwd[user] = pass
\end{minted}
\end{frame}

\begin{frame}[fragile]
  \frametitle{Operacije nad rečnikom}
\begin{center}
\begin{tabular}{l|l}
\textbf{operacija} & \textbf{značenje} \\ \hline
\texttt{<dict>.has\_key(<key>)} & vraća \texttt{True} ako postoji dati ključ \\
\texttt{<dict>.keys()} & vraća listu ključeva \\
\texttt{<dict>.values()} & vraća listu vrednosti \\
\texttt{<dict>.items()} & vraća listu sekvenci \texttt{(key,val)} \\
\texttt{del <dict>[<key>]} & uklanja element iz rečnika \\
\texttt{<dict>.clear()} & uklanja sve elemente iz rečnika
\end{tabular}
\end{center}
\end{frame}

\section[Primer]{Primer: brojanje reči}

\begin{frame}[fragile]
  \frametitle{Primer: brojanje reči u tekstu}
  \begin{itemize}
    \item program analizira tekst i prebrojava koliko se puta svaka reč pojavila u tekstu
    \item ignorišemo različite gramatičke oblike (posmatramo ih kao različite reči) \\ \ \\
    \item možemo rešavati problem pomoću akumulatora
    \item samo nam treba akumulator za svaku reč koju pronađemo u tekstu
    \item ne znamo unapred koje ćemo reči pronaći
    \item nema smisla inicijalizovati akumulator za svaku moguću reč \\ \ \\
    \item kako upotrebiti rečnik?
  \end{itemize}
\end{frame}

\begin{frame}[fragile]
  \frametitle{Primer: brojanje reči u tekstu $_2$}
  \begin{itemize}
    \item ključevi su stringovi sa rečima
    \item vrednosti su brojevi koliko se puta reč pojavila u tekstu
    \item kada naiđemo na reč u tekstu, inkrementiramo njen akumulator:
  \end{itemize}
\begin{verbatim}
counts[word] = counts[word] + 1
\end{verbatim}
\end{frame}

\begin{frame}[fragile]
  \frametitle{Primer: brojanje reči u tekstu $_3$}
\begin{minted}{python}
counts[word] = counts[word] + 1
\end{minted}
  \begin{itemize}
    \item šta kada prvi put naiđemo na reč u tekstu?
    \item \texttt{counts[word]} ne postoji, dobićemo grešku
    \item $\Rightarrow$ nova reč nema svoj element u rečniku
    \item kada naiđemo na novu reč, moramo dodati novi element u rečnik
  \end{itemize}
\end{frame}

\begin{frame}[fragile]
  \frametitle{Dodavanje novog elementa u rečnik}
  \begin{itemize}
    \item varijanta 1: prvo proverimo da li reč postoji u rečniku
  \end{itemize}
\begin{minted}{python}
if counts.has_key(word):
    counts[word] = counts[word] + 1
else:
    counts[word] = 1
\end{minted}
  \begin{itemize}
    \item varijanta 2: pokušamo da inkrementiramo akumulator;
    ako pukne, dodamo novi element u rečnik
  \end{itemize}
\begin{minted}{python}
try:
    counts[word] = counts[word] + 1
except KeyError:
    counts[word] = 1
\end{minted}
\end{frame}

\begin{frame}[fragile]
  \frametitle{Učitavanje teksta i podela na reči}
\begin{minted}{python}
fname = input("Ime fajla: ")

# učitaj fajl kao dugačak string
text = open(fname, "r").read()

# pretvori sva slova u mala
text = string.lower(text)

# zameni svaki znak interpunkcije razmakom
for ch in "!\"#$%&()*+,-./:;<=>?@[\\]ˆ_'{|}":
    text = text.replace(ch, " ")

# podeli string na reči
words = string.split(text)
\end{minted}
\end{frame}

\begin{frame}[fragile]
  \frametitle{Brojanje parsiranih reči}
\begin{minted}{python}
counts = {}
for w in words:
    try:
        counts[w] = counts[w] + 1
    except KeyError:
        counts[w] = 1
\end{minted}
\end{frame}

\begin{frame}[fragile]
  \frametitle{Ispis rezultata}
\begin{minted}{python}
# uzmi listu reči
uniqueWords = counts.keys()

# sortiraj listu reči
uniqueWords.sort()

# ispiši svaku reč i broj pojavljivanja
for w in uniqueWords:
    print(w, counts[w])
\end{minted}
\end{frame}

\begin{frame}[fragile]
  \frametitle{Sortiranje}
\begin{verbatim}
uniqueWords.sort()
\end{verbatim}
  \begin{itemize}
    \item reči će nam biti sortirane po abecedi
    \item kako da sortiramo reči po broju pojavljivanja?
    \item \texttt{sort} može da primi parametar -- funkciju za poređenje
  \end{itemize}
\begin{verbatim}
uniqueWords.sort(compareWords)
\end{verbatim}
\end{frame}

\begin{frame}[fragile]
  \frametitle{Sortiranje $_2$}
\begin{minted}{python}
def compareWords((w1,c1), (w2,c2)):
    if c1 > c2:
        return -1
    elif c1 == c2:
        return cmp(w1, w2)
    else:
        return 1
\end{minted}
  \begin{itemize}
    \item funkcija vraća -1 ako je prva reč ,,češća`` od druge
    \item vraća +1 ako je prva reč ,,ređa`` od druge
    \item ako se reči pojavljuju jednako često, tada ih uporedi po abecedi
  \end{itemize}
\end{frame}

\begin{frame}[fragile]
  \frametitle{Sortiranje $_3$}
\begin{minted}{python}
uniqueWords.sort(compareWords)
\end{minted}
  \begin{itemize}
    \item funkcija \texttt{compareWords} se koristi kao parametar
    \item nema zagrada -- \texttt{compareWords} se \myblue{ne poziva}!
    \item ova funkcija, data kao parametar funkciji \texttt{sort}, biće pozvana unutar \texttt{sort}
  \end{itemize}
\end{frame}

\begin{frame}[fragile]
  \frametitle{Ispis rezultata}
  \begin{itemize}
    \item ispisujemo prvih \texttt{n} najčešćih reči
  \end{itemize}
\begin{minted}{python}
for i in range(n):
    print("{0:<10}{1:>6}".format(uniqueWords[i],
        counts[uniquewords[i]])
\end{minted}
\end{frame}


\end{document}