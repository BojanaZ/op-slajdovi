\documentclass[utf8,compress]{beamer}
\usepackage[utf8]{inputenc}
\usepackage[T1]{fontenc}
\usepackage{irbookslide}
\usepackage{irilmenau2}

\title{Rukovanje brojevima}
\subtitle{\tiny{Slajdovi za predmet Osnove programiranja}}
\subject{Osnove programiranja}
\institute{Katedra za informatiku, Fakultet tehničkih nauka, Novi Sad}
\date{2011.}

\begin{document}

\frame{\titlepage}

\frame{
  \frametitle{Ciljevi}
  \begin{itemize}
    \item razumevanje pojma tipova podataka
    \item poznavanje osnovnih numeričkih tipova podataka u Pythonu
    \item razumevanje osnovnih principa reprezentacije brojeva u računaru
    \item korišćenje Pythonove \texttt{math} biblioteke
    \item razumevanje akumulator šablona
    \item razumevanje i pisanje programa koji operišu numeričkim podacima
  \end{itemize}
}

\section{Tipovi podataka}

\frame{
  \frametitle{Numerički tipovi podataka}
  \begin{itemize}
    \item informacije koje obrađuje računar su predstavljene u obliku podataka
    \item dve vrste brojeva:
    \begin{itemize}
      \item \myblue{celi brojevi} -- nemaju razlomljeni deo, npr. 5, 4, 3, 6
      \item \myblue{decimalni ostaci} -- brojevi između 0 i 1, npr. .25, .1, .05, .01
    \end{itemize}
  \end{itemize}
}

\frame{
  \frametitle{Numerički tipovi podataka $_2$}
  \begin{itemize}
    \item celi i razlomljeni brojevi su različito predstavljeni unutar računara!
    \item kažemo da su celi i razlomljeni brojevi različitog \myblue{tipa}
    \item tip podatka određuje koje vrednosti promenljiva može imati i koje operacije možemo izvršiti nad njima
  \end{itemize}
}

\frame{
  \frametitle{Numerički tipovi podataka $_3$}
  \begin{itemize}
    \item celi brojevi se predstavljaju tipom podataka \myblue{integer} (kraće \myblue{int})
    \item vrednosti tipa \myblue{int} mogu biti pozitivni i negativni celi brojevi i nula
  \end{itemize}
}

\frame{
  \frametitle{Numerički tipovi podataka $_4$}
  \begin{itemize}
    \item brojevi koji imaju razlomljeni deo predstavljaju se kao \myblue{brojevi u pokretnom zarezu} (\myblue{float})
    \item kako ih razlikujemo?
    \begin{itemize}
      \item numerički literal bez decimalne tačne predstavlja \texttt{int} vrednost
      \item numerički literal sa decimalnom tačkom predstavlja \texttt{float} vrednost (makar sve decimale bile 0)
    \end{itemize}
  \end{itemize}
}

\begin{frame}[fragile]
  \frametitle{Numerički tipovi podataka $_5$}
  \begin{itemize}
    \item Python ima posebnu funkciju \texttt{type} koja vraća tip date vrednosti
  \end{itemize}
\begin{verbatim}
>>> type(3)
<class 'int'>
>>> type(3.1)
<class 'float'>
>>> type(3.0)
<class 'float'>
>>> myInt = 32
>>> type(myInt)
<class 'int'>
>>>
\end{verbatim}
\end{frame}

\frame{
  \frametitle{Numerički tipovi podataka $_6$}
  \begin{itemize}
    \item zašto nam trebaju dve vrste brojeva?
    \begin{itemize}
      \item brojači ne mogu biti razlomljeni
      \item puno matematičkih algoritama su vrlo efikasni sa celim brojevima
      \item tip \texttt{float} čuva samo \myblue{aproksimaciju} realnog broja
      \item pošto \texttt{float} brojevi nisu egzaktni, treba koristiti \texttt{int} kad god je moguće
    \end{itemize}
  \end{itemize}
}

\begin{frame}[fragile,shrink=5]
  \frametitle{Numerički tipovi podataka $_7$}
  \begin{itemize}
    \item operacije nad \texttt{int}-ovima proizvode \texttt{int}-ove, osim \texttt{/}
    \item operacije nad \texttt{float}-ovima proizvode \texttt{float}-ove
  \end{itemize}
\begin{verbatim}
>>> 3.0+4.0
7.0
>>> 3+4
7
>>> 3.0*4.0
12.0
>>> 3*4
12
>>> 10.0/3.0
3.3333333333333335
>>> 10/3
3.3333333333333335
>>> 10 // 3
3
>>> 10.0 // 3.0
3.0
\end{verbatim}
\end{frame}

\frame{
  \frametitle{Numerički tipovi podataka $_8$}
  \begin{itemize}
    \item celobrojno deljenje proizvodi ceo broj: \texttt{10//3 = 3}
    \item ostatak pri deljenju: \texttt{10\%3 = 1}
    \item \texttt{a = (a//b)*b + a\%b}
  \end{itemize}
}

\section{Paket \texttt{math}}

\frame{
  \frametitle{Paket \texttt{math}}
  \begin{itemize}
    \item pored aritmetičkih operacija (+,-,*,/,//,**,\%,abs) postoji još puno matematičkih funkcija u paketu \texttt{math}
    \item paket je skup korisnih funkcija
  \end{itemize}
}

\frame{
  \frametitle{Korišćenje \texttt{math} paketa}
  \begin{itemize}
    \item napišimo program koji izračunava korene kvadratne jednačine \\
    $$ x = \frac{-b \pm \sqrt{b^2-4ac}}{2a}$$
    \item jedini deo koji nam nedostaje je izračunavanje kvadratnog korena...
    \item za to postoji funkcija u paketu \texttt{math}
  \end{itemize}
}

\frame{
  \frametitle{Korišćenje \texttt{math} paketa $_2$}
  \begin{itemize}
    \item da bismo koristili paket, potreban nam je sledeći red u programu: \\
      \texttt{import math}
    \item \texttt{import}-i obično stoje na početku programa
    \item importovanje paketa čini funkcije iz tog paketa dostupnima u našem programu
  \end{itemize}
}

\frame{
  \frametitle{Korišćenje \texttt{math} paketa $_3$}
  \begin{itemize}
    \item funkciju \texttt{sqrt} pozivamo sa \texttt{math.sqrt(x)}
    \item ,,tačka-notacija`` kaže Pythonu da funkciju \texttt{sqrt} traži u paketu
    \item za izračunavanje korena možemo pisati: \\
      \texttt{root = math.sqrt(b*b - 4*a*c)}
  \end{itemize}
}

\section{Primer 1}

\begin{frame}[fragile,shrink=5]
  \frametitle{Korišćenje \texttt{math} paketa $_4$}
\begin{verbatim}
# quadratic.py
#    Izračunava realne korene kvadratne jednačine
#    Ilustruje korišćenje math paketa
#    Pažnja: program puca ako nema realnih korena

import math  # čini funkcije iz paketa dostupnim

print("Izračunavanje korena kvadratne jednačine")
print()

a, b, c = eval(input("Unesite koeficijente (a, b, c): "))

root = math.sqrt(b * b - 4 * a * c)
root1 = (-b + root) / (2 * a)
root2 = (-b - root) / (2 * a)

print()
print("Rešenja su:", root1, root2)
\end{verbatim}
\end{frame}

\begin{frame}[fragile,shrink=5]
  \frametitle{Korišćenje \texttt{math} paketa $_5$}
\begin{verbatim}
$ python quadratic.py
Izračunavanje korena kvadratne jednačine

Unesite koeficijente (a, b, c): 3, 4, -1

Rešenja su: 0.215250437022 -1.54858377035
\end{verbatim}
  \begin{itemize}
    \item Šta je ovo?
  \end{itemize}
\begin{verbatim}
$ python quadratic.py
Izračunavanje korena kvadratne jednačine

Unesite koeficijente (a, b, c): 1, 2, 3

Traceback (most recent call last):
  File "/home/branko/pajton/quadratic.py", line 14, in -toplevel-
    root = math.sqrt(b * b - 4 * a * c)
ValueError: math domain error
\end{verbatim}

\end{frame}

\frame{
  \frametitle{Korišćenje \texttt{math} paketa $_6$}
  \begin{itemize}
    \item ako je $a=1, b=2, c=3$, pokušavamo da izračunamo koren negativnog broja
    \item umesto \texttt{math.sqrt} možemo da upotrebimo \texttt{**}
    \item kako iskoristiti \texttt{**} za izračunavanje korena?
  \end{itemize}
}

\section{Akumulatori}

\frame{
  \frametitle{Akumuliranje rezultata: faktorijel}
  \begin{itemize}
    \item čekaš u redu sa još 5 ljudi; na koliko načina se može poređati 6 ljudi?
    \item permutacije 6 elemenata
    \item ukupan broj permutacija = 6! (6 faktorijel) = 720
    \item faktorijel se definiše ovako: $n! = n(n-1)(n-2)...(1)$
    \item prema tome $6! = 6*5*4*3*2*1 = 720$
  \end{itemize}
}

\frame{
  \frametitle{Akumuliranje rezultata: faktorijel $_2$}
  \begin{itemize}
    \item kako napisati program za računanje faktorijela?
    \item[1] uneti broj čiji faktorijel se računa
    \item[2] izračunati faktorijel
    \item[3] ispisati rezultat
  \end{itemize}
}

\frame{
  \frametitle{Akumuliranje rezultata: faktorijel $_3$}
  \begin{itemize}
    \item kako smo izračunali 6! ?
    \item[1] 6*5 = 30
    \item[2] uzeti tih 30, i pomnožiti 30 * 4 = 120
    \item[3] uzeti tih 120, i pomnožiti 120 * 3 = 360
    \item[4] uzeti tih 360, i pomnožiti 360 * 2 = 720
    \item[5] uzeti tih 720, i pomnožiti 720 * 1 = 720
  \end{itemize}
}

\frame{
  \frametitle{Akumuliranje rezultata: faktorijel $_4$}
  \begin{itemize}
    \item ponavljamo množenje i pamtimo međurezultat
    \item ovakvi algoritmi se zovu \myblue{akumulatori}, jer postepeno gradimo (akumuliramo) rezultat
    \item međurezultati i konačni rezultat se čuvaju u \myblue{akumulatorskoj promenljivoj}
  \end{itemize}
}

\frame{
  \frametitle{Akumuliranje rezultata: faktorijel $_5$}
  \begin{itemize}
    \item opšti oblik akumulatorskog algorima izgleda ovako:
    \item[1] inicijalizuj akumulatorsku promenljivu
    \item[2] ponavljaj narednu operaciju dok se ne dobije konačan rezultat
    \item[2a] \ \ \ \ ažuriraj vrednost akumulatorske promenljive
    \item[3] rezultat je u akumulatorskoj promenljivoj
  \end{itemize}
}

\begin{frame}[fragile]
  \frametitle{Akumuliranje rezultata: faktorijel $_6$}
  \begin{itemize}
    \item trebaće nam petlja
  \end{itemize}
\begin{verbatim}
fact = 1
for i in (6,5,4,3,2,1):
    fact = fact * i
\end{verbatim}
  \begin{itemize}
    \item da probamo ovo ručno da izvršimo
  \end{itemize}
\end{frame}

\frame{
  \frametitle{Akumuliranje rezultata: faktorijel $_7$}
  \begin{itemize}
    \item zašto smo inicijalizovali \texttt{fact} na 1?
    \begin{itemize}
      \item u svakom ciklusu petlje prethodna vrednost \texttt{fact} se koristi za računanje sledeće
      \item u prvom prolazu \texttt{fact} će imati vrednost zahvaljujući inicijalizaciji
    \end{itemize}
    \item ako bismo izostavili inicijalizaciju šta bi se desilo?
  \end{itemize}
}

\begin{frame}[fragile]
  \frametitle{Akumuliranje rezultata: faktorijel $_8$}
  \begin{itemize}
    \item množenje je komutativno, možemo napisati program ovako:
  \end{itemize}
\begin{verbatim}
fact = 1
for i in (2,3,4,5,6):
    fact = fact * i
\end{verbatim}
  \begin{itemize}
    \item šta ako želimo faktorijel nekog drugog broja?
  \end{itemize}
\end{frame}

\frame{
  \frametitle{Akumuliranje rezultata: faktorijel $_9$}
  \begin{itemize}
    \item šta vraća \texttt{range(n)}? \\
      0, 1, 2, 3, ..., n-1
    \item \texttt{range} može imati još parametara; \texttt{range(start, n)} vraća \\
      start, start+1, ..., n-1
    \item \texttt{range(start, n, step)} vraća \\
      start, start+step, ..., n-1
  \end{itemize}
}

\frame{
  \frametitle{Akumuliranje rezultata: faktorijel $_{10}$}
  \begin{itemize}
    \item pomoću \texttt{range} funkcije možemo napraviti našu \texttt{for} petlju na više načina
    \item možemo brojati od 2 do n: \\
      \texttt{range(2, n+1)} \ \ \ \ (zašto n+1?)
    \item možemo brojati od n do 2: \\
      \texttt{range(n, 1, -1)}
  \end{itemize}
}

\section{Primer 2}


\begin{frame}[fragile]
  \frametitle{Akumuliranje rezultata: faktorijel $_{11}$}
  \begin{itemize}
    \item kompletan program za računanje faktorijela
  \end{itemize}
\begin{verbatim}
# factorial.py
#    Program računa faktorijel datog broja
#    Ilustruje petlju sa akumulatorom

n = eval(input("Unesite prirodan broj: "))
fact = 1
for i in range(n,1,-1): 
    fact = fact * i
print("Faktorijel od", n, "je", fact)
\end{verbatim}
\end{frame}

\begin{frame}[fragile]
  \frametitle{Granice \texttt{int}-a}
  \begin{itemize}
    \item koliko je 100! ?
  \end{itemize}
\begin{verbatim}
$ python factorial.py
Unesite prirodan broj: 100
Faktorijel od 100 je 93326215443944152681699238856266700
49071596826438162146859296389521759999322991560894146397
61565182862536979208272237582511852109168640000000000000
00000000000
\end{verbatim}
  \begin{itemize}
    \item prilično veliki broj
  \end{itemize}
\end{frame}

\section{Opseg brojeva}

\frame{
  \frametitle{Granice \texttt{int}-a $_2$}
  \begin{itemize}
    \item novije verzije Pythona mogu da izračunaju 100!
    \item u starijim verzijama ovaj program će pući već za n=13
    \item \texttt{OverflowError: integer multiplication}
  \end{itemize}
}

\frame{
  \frametitle{Granice \texttt{int}-a $_3$}
  \begin{itemize}
    \item iako postoji beskonačno mnogo celih brojeva, samo konačan broj njih se može predstaviti pomoću \texttt{int}-a
    \item opseg mogućih vrednosti zavisi od broja \myblue{bita} koje CPU koristi za predstavljanje \texttt{int}-a
    \item tipično se za \texttt{int} koristi 32 bita
    \item<2> to znači da ima $2^{32}$ mogućih vrednosti, sa 0 u sredini
    \item<2> opseg je od $-2^{31}$ do $2^{31}-1$ \\(ovo -1 je zbog nule koja takođe zauzima mesto)
    \item<2> 100! je mnogo veće od $2^{31}$; kako to radi?
  \end{itemize}
}

\begin{frame}[fragile]
  \frametitle{Rukovanje velikim brojevima}
  \begin{itemize}
    \item ako koristimo \texttt{float} umesto \texttt{int}-a?
    \item kada inicijalizujemo akumulator na 1.0 dobijamo
  \end{itemize}
\begin{verbatim}
$ python factorial.py
Unesite prirodan broj: 15
Faktorijel od 15 je 1.307674368e+012
\end{verbatim}
  \begin{itemize}
    \item više ne dobijamo potpuno tačan rezultat!
  \end{itemize}
\end{frame}

\frame{
  \frametitle{Rukovanje velikim brojevima $_2$}
  \begin{itemize}
    \item vrlo veliki ili vrlo mali brojevi se izražavaju u \myblue{eksponencijalnoj notaciji}
    \item \texttt{1.307674368e+012} predstavlja broj $1.307674368 \cdot 10^{12}$
    \item ovim množenjem se decimalna tačka pomera za 12 mesta u desno
    \item ali broj ima 9 decimala
    \item 3 cifre su izgubljene!
  \end{itemize}
}

\frame{
  \frametitle{Rukovanje velikim brojevima $_3$}
  \begin{itemize}
    \item \texttt{float}-ovi su aproksimacije
    \item pomoću njih možemo prikazati veći opseg brojeva, ali sa manjom preciznošću
    \item (noviji) Python ima rešenje -- prošireni \texttt{int}-ovi
    \item fleksibilnost na račun brzine i potrošnje memorije
  \end{itemize}
}

\section{Konverzije tipova}

\frame{
  \frametitle{Konverzije tipova}
  \begin{itemize}
    \item kombinovanje \texttt{int}-a i \texttt{int}-a proizvodi \texttt{int}
    \item kombinovanje \texttt{float}-a i \texttt{float}-a proizvodi \texttt{float}
    \item šta se dešava kada ih pomešamo? \\
      \texttt{x = 5.0 + 2}
    \item šta bi trebalo da bude rezultat?
  \end{itemize}
}

\frame{
  \frametitle{Konverzije tipova $_2$}
  \begin{itemize}
    \item morali bismo ili konvertovati 5.0 u 5 i obaviti \texttt{int}-sabiranje
    \item ili konvertovati 2 u 2.0 i obaviti \texttt{float}-sabiranje 
    \item kovertovanje \texttt{float} u \texttt{int} (u opštem slučaju) nosi gubitak podataka
    \item \texttt{int}-ovi se lako konvertuju u \texttt{float} dodavanjem .0
  \end{itemize}
}

\frame{
  \frametitle{Konverzije tipova $_3$}
  \begin{itemize}
    \item u \myblue{mešovitim} izrazima Python će automatski konvertovati \\ \texttt{int} $\rightarrow$ \texttt{float}
    \item nekada mi želimo da kontrolišemo konverziju tipa \\ -- \myblue{eksplicitna tipizacija}
  \end{itemize}
}

\begin{frame}[fragile]
  \frametitle{Konverzije tipova $_4$}
\begin{verbatim}
>>> float(22//5)
4.0
>>> int(4.5)
4
>>> int(3.9)
3
>>> round(3.9)
4
>>> round(3)
3
\end{verbatim}
\end{frame}


\end{document}