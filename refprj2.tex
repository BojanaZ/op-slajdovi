\documentclass[a4paper]{article}
\usepackage[utf8]{inputenc}
\usepackage[T1]{fontenc}
\usepackage{minted}

\frenchspacing
\setlength{\hoffset}{0cm}
\setlength{\voffset}{0cm}
\setlength{\textwidth}{16cm}
\setlength{\textheight}{25cm}
\setlength{\oddsidemargin}{0cm}
\setlength{\topmargin}{0cm}
\setlength{\headheight}{0cm}
\setlength{\headsep}{0cm}
\setlength{\abovecaptionskip}{3pt}
\setlength{\belowcaptionskip}{0pt}

\title{Primer projektnog zadatka: studentska služba}
\date{}

\begin{document}

\maketitle

\section{Definicija projektnog zadatka}

Implementirati aplikaciju za vođenje evidencije u studentskoj službi. Aplikacija
treba da obezbedi rad sa sledećim entitetima:

\begin{itemize}
  \item \textit{student} opisan \textit{brojem indeksa}, \textit{imenom},
  \textit{prezimenom}, \textit{imenom roditelja},
  \textit{datumom rođenja}, \textit{JMBG brojem}, \textit{adresom},
  \textit{telefonom}, \textit{e-mailom} i \textit{tekućom godinom studija}. Pri
  tome ne mogu postojati dva studenta sa identičnim brojem indeksa niti JMBG
  brojem. Fajl sa spiskom studenata se ažurira iz aplikacije, nakon izvršavanja
  odgovarajućih komandi.

  \item \textit{Referent} studentske službe opisan \textit{imenom},
  \textit{prezimenom}, \textit{korisničkim imenom} i \textit{lozinkom}.
\end{itemize}

Izbor formata u kome se podaci o studentima i referentima zapisuju u fajlove je
deo projektnog zadatka. Nakon pokretanja aplikacije referent ima mogućnost
jedino da se prijavi na sistem, pri čemu unosi korisničko ime i lozinku. Nakon
uspešnog prijavljivanja na sistem aplikacija referentu treba da omogući sledeće:

\begin{itemize}
  \item Pronalaženje studenta po broju indeksa. Nakon izvršavanja ove komande
  prikazuju se podaci o studentu sa datim brojem indeksa.

  \item Pretraživanje studenata po prezimenu. Nakon izvršavanja svake od ovih
  komandi prikazuju se podaci o studentima koji zadovoljavaju kriterijum
  pretraživanja.

  \item Pregledanje svih studenata sortiranih po prezimenu.

  \item Izmenu adrese, telefona i e-maila studenta. Pri izvršavanju ove komande,
  referent unosi broj indeksa studenta čiji podaci se menjaju i potom nove 
  podatke.
  
  \item Upis studenata na prvu godinu studija, tj. upis novog studenta koji 
  prethodno nije bio evidentiran u sistemu.
  
  \item Grupni upis studenata na narednu godinu studija. Pri tome, unose se
  brojevi indeksa onih studenata koji upisuju narednu godinu, i podatak o
tekućoj
  godini studija se inkrementira za svakog od njih.
  
\end{itemize}

\section{Rešenje zadatka}

\subsection{Rukovanje referentima}

Jednog referenta studentske službe ćemo u programu reprezentovati rečnikom, čiji
elementi odgovaraju podacima koje evidentiramo za svakog referenta. Sledeći
primer predstavlja opis jednog referenta studentske službe.

\begin{listing}[htb]
\setlength\partopsep{-\topsep}
\addtolength\partopsep{-\parskip}
\begin{minted}{python}
referent = {
  'ime': 'Mitar',
  'prezime': 'Mirić',
  'username': 'mitar',
  'password': '123'}
\end{minted}
\caption{Python listing}
\label{lst:java1}
\end{listing}

\end{document}

