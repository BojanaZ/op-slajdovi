\documentclass[utf8,compress]{beamer}
\usepackage{irbookslide}
\usepackage{irilmenau2}
\usepackage{url}
\usepackage{fontspec} % zahteva paket euenc
\usepackage{xunicode}
\usepackage{xltxtra}
\usepackage{polyglossia}
\usepackage{minted}
\usepackage{xcolor,colortbl}
\usepackage{textcomp}
\usepackage{unicode-math}

\title{Sekvence: stringovi, liste, fajlovi}
\subtitle{\tiny{Slajdovi za predmet Osnove programiranja}}
\subject{Osnove programiranja}
\institute{Katedra za informatiku, Fakultet tehničkih nauka, Novi Sad}
\date{2014.}

\begin{document}

% da pygmentize ne uokviruje crvenom bojom nepoznate karaktere
\expandafter\def\csname PY@tok@err\endcsname{}

\frame{\titlepage}

\frame{
  \frametitle{Ciljevi}
  \begin{itemize}
    \item razumevanje stringova i njihove reprezentacije u računaru
    \item poznavanje operacija nad stringovima
    \item poznavanje pojma sekvence i indeksa, primena na stringove i liste
    \item veština formatiranja stringova
  \end{itemize}
}

\frame{
  \frametitle{Ciljevi $_2$}
  \begin{itemize}
    \item poznavanje osnovnih koncepata rada sa fajlovima
    \item poznavanje tehnika za čitanje i pisanje fajlova u Pythonu
    \item razumevanje i pisanje programa koji rukuju tekstualnim fajlovima
  \end{itemize}
}

\section{String}

\frame{
  \frametitle{Tip podataka string}
  \begin{itemize}
    \item obrada teksta je najčešći oblik korišćenja računara
    \item u programima tekst je predstavljen tipom \myblue{string}
    \item string je niz (sekvenca) znakova
    \item navodi se unutar jednostrukih (') ili dvostrukih (") navodnika
  \end{itemize}
}

\begin{frame}[fragile]
  \frametitle{Tip podataka string $_2$}
\begin{minted}[linenos=false]{python}
>>> str1="Hello"
>>> str2='spam'
>>> print(str1, str2)
Hello spam
>>> type(str1)
<class 'str'>
>>> type(str2)
<class 'str'>
\end{minted}
\end{frame}

\begin{frame}[fragile]
  \frametitle{Tip podataka string $_3$}
  \begin{itemize}
    \item unos stringa sa tastature
  \end{itemize}
\begin{minted}[linenos=false]{python}
>>> firstName = input("Please enter your name: ")
Please enter your name: Žika
>>> print("Hello", firstName)
Hello Žika
\end{minted}
  \begin{itemize}
    \item uneti string nije \texttt{eval}-uiran -- želimo da memorišemo unete znakove, a ne da ih interpretiramo kao Python izraz
  \end{itemize}
\end{frame}

\frame{
  \frametitle{String i pojedinačni znakovi}
  \begin{itemize}
    \item možemo pristupiti pojedinačnim znakovima u stringu pomoću \myblue{indeksa}
    \item indeks predstavlja redni broj znaka u stringu
    \item brojanje počinje od 0
    \item opšti oblik je \texttt{<string>[<expr>]} gde vrednost \texttt{expr} određuje izabrani znak u stringu
  \end{itemize}
}

\begin{frame}[fragile]
  \frametitle{String i pojedinačni znakovi $_2$}
\begin{tabular}{|c|c|c|c|c|c|c|c|c|}
\cline{1-9}
H & e & l & l & o & \  & B & o & b \\ \cline{1-9}
\multicolumn{1}{c}{0} & \multicolumn{1}{c}{1} & \multicolumn{1}{c}{2} & \multicolumn{1}{c}{3} & \multicolumn{1}{c}{4} & \multicolumn{1}{c}{5} & \multicolumn{1}{c}{6} & \multicolumn{1}{c}{7} & \multicolumn{1}{c}{8}
\end{tabular}
\begin{minted}[linenos=false]{python}
>>> greet = "Hello Bob"
>>> greet[0]
'H'
>>> print(greet[0], greet[2], greet[4])
H l o
>>> x = 8
>>> print(greet[x - 2])
B
\end{minted}
\end{frame}

\begin{frame}[fragile]
  \frametitle{String i pojedinačni znakovi $_3$}
\begin{tabular}{|c|c|c|c|c|c|c|c|c|}
\cline{1-9}
H & e & l & l & o & \  & B & o & b \\ \cline{1-9}
\multicolumn{1}{c}{0} & \multicolumn{1}{c}{1} & \multicolumn{1}{c}{2} & \multicolumn{1}{c}{3} & \multicolumn{1}{c}{4} & \multicolumn{1}{c}{5} & \multicolumn{1}{c}{6} & \multicolumn{1}{c}{7} & \multicolumn{1}{c}{8}
\end{tabular}
\begin{itemize}
  \item u stringu od $n$ znakova, poslednji znak je na poziciji $n-1$ jer brojanje počinjemo od $0$
  \item možemo brojati i sa desnog kraja pomoću negativnih brojeva
\end{itemize}
\begin{minted}[linenos=false]{python}
>>> greet[-1]
'b'
>>> greet[-3]
'B'
\end{minted}
\end{frame}

\frame{
  \frametitle{String i pojedinačni znakovi $_4$}
  \begin{itemize}
    \item indeksiranje vraća string koji sadrži tačno jedan znak
    \item možemo izdvojiti i duži podniz znakova iz stringa -- \myblue{podstring} (substring)
    \item postupak \myblue{isecanja} (slicing)
  \end{itemize}
}

\frame{
  \frametitle{String i pojedinačni znakovi $_5$}
  \begin{itemize}
    \item isecanje: \\
      \texttt{<string>[<start>:<end>]}
    \item \texttt{start} i \texttt{end} moraju biti tipa \texttt{int}
    \item isečak (slice) sadrži podstring od \texttt{start}-nog znaka (uključivo) do \texttt{end}-nog (isključivo)
  \end{itemize}
}

\begin{frame}[fragile]
  \frametitle{String i pojedinačni znakovi $_6$}
\begin{tabular}{|c|c|c|c|c|c|c|c|c|}
\cline{1-9}
H & e & l & l & o & \  & B & o & b \\ \cline{1-9}
\multicolumn{1}{c}{0} & \multicolumn{1}{c}{1} & \multicolumn{1}{c}{2} & \multicolumn{1}{c}{3} & \multicolumn{1}{c}{4} & \multicolumn{1}{c}{5} & \multicolumn{1}{c}{6} & \multicolumn{1}{c}{7} & \multicolumn{1}{c}{8}
\end{tabular}
\begin{minted}[linenos=false]{python}
>>> greet[0:3]
'Hel'
>>> greet[5:9]
' Bob'
>>> greet[:5]
'Hello'
>>> greet[5:]
' Bob'
>>> greet[:]
'Hello Bob'
\end{minted}
  \begin{itemize}
    \item ako nedostaju \texttt{start} ili \texttt{end} podrazumevaju se početak odnosno kraj stringa
  \end{itemize}
\end{frame}

\frame{
  \frametitle{Konkatenacija i ponavljanje stringa}
  \begin{itemize}
    \item \myblue{konkatenacija} predstavlja ,,spajanje`` stringova -- kraj prvog stringa sa početkom drugog
    \item operator za konkatenaciju je + \\\ %
    \item \myblue{ponavljanje} predstavlja višestruku konkatenaciju stringa sa samim sobom
    \item operator za ponavljanje je *
  \end{itemize}
}

\begin{frame}[fragile]
  \frametitle{Konkatenacija i ponavljanje stringa $_2$}
\begin{verbatim}
>>> "spam" + "eggs"
'spameggs'
>>> "Spam" + "And" + "Eggs"
'SpamAndEggs'
>>> 3 * "spam"
'spamspamspam'
>>> "spam" * 5
'spamspamspamspamspam'
>>> (3 * "spam") + ("eggs" * 5)
'spamspamspameggseggseggseggseggs'
\end{verbatim}
\end{frame}

\begin{frame}[fragile]
  \frametitle{Dužina stringa}
  \begin{itemize}
    \item funkcija \texttt{len} vraća dužinu (broj znakova) stringa
  \end{itemize}
\begin{minted}[linenos=false]{python}
>>> len("spam")
4
>>> for ch in "Spam!":
        print (ch, end=" ")
	
S p a m !
\end{minted}
\end{frame}

\begin{frame}[fragile]
  \frametitle{Operacije nad stringom}
\begin{center}
\begin{tabular}{l|l}
\textbf{operator} & \textbf{značenje} \\ \hline
\texttt{+} & konkatenacija \\
\texttt{*} & ponavljanje \\
\texttt{<string>[]} & indeksiranje \\
\texttt{<string>[:]} & isecanje \\
\texttt{for <var> in <string>} & iteracija kroz znakove
\end{tabular}
\end{center}
\end{frame}

\section{Primeri}

\begin{frame}[fragile]
  \frametitle{Primeri obrade stringova}
  \begin{itemize}
    \item korisničko ime u nekom računarskom sistemu: \\
      prvo slovo imena i prvih 7 slova prezimena
  \end{itemize}
\begin{minted}[linenos=false]{python}
first = input("Unesite ime: ")
last = input("Unesite prezime: ")

username = first[0] + last[:7]
\end{minted}
\end{frame}

\begin{frame}[fragile]
  \frametitle{Formiranje korisničkog imena}
\begin{verbatim}
Unesite ime: ana
Unesite prezime: tot
username = atot

Unesite ime: dimitrije
Unesite prezime: dimitrijević
username = ddimitri
\end{verbatim}
\end{frame}

\frame{
  \frametitle{Redni broj meseca u godini $\rightarrow$ troslovna oznaka}
  \begin{itemize}
    \item smestimo sve troslovne oznake u jedan veliki string: \\
      \texttt{"JanFebMarAprMajJunJulAvgSepOktNovDec"}
    \item koristimo redni broj meseca kao indeks za string: \\
      \texttt{mesecSkr = meseci[pos:pos+3]}
    \item da bismo izračunali \texttt{pos}, treba od broja meseca oduzeti 1 i pomnožiti to sa 3
  \end{itemize}
\begin{center}
\begin{tabular}{l|r|r}
\textbf{mesec} & \textbf{broj} & \texttt{pos} \\ \hline
Jan & 1 & 0 \\
Feb & 2 & 3 \\
Mar & 3 & 6 \\
Apr & 4 & 9 \\
\end{tabular}
\end{center}}

\begin{frame}[fragile,shrink=10]
  \frametitle{Redni broj meseca u godini $\rightarrow$ troslovna oznaka $_2$}
\begin{minted}[linenos=false]{python}
# month.py
# Štampa skraćeno ime meseca za dati redni broj

# ovde čuvamo nazive
meseci = "JanFebMarAprMajJunJulAvgSepOktNovDec"

n = eval(raw_input("Unesite mesec (1-12): "))

# izračunaj indeks u stringu
pos = (n-1) * 3

# iseci podstring
mesecSkr = meseci[pos:pos+3]

# ispiši rezultat
print ("Skraćenica je", mesecSkr + ".")
\end{minted}
\end{frame}

\begin{frame}[fragile]
  \frametitle{Redni broj meseca u godini $\rightarrow$ troslovna oznaka $_3$}
\begin{verbatim}
$ python month.py
Unesite mesec (1-12): 1
Skraćenica je Jan.
$ python month.py
Unesite mesec (1-12): 12
Skraćenica je Dec.
\end{verbatim}
  \begin{itemize}
    \item ovo rešenje je ispravno samo ako su nazivi svih meseci iste dužine
    \item kako da to prevaziđemo?
  \end{itemize}
\end{frame}

\section{Liste}

\begin{frame}[fragile]
  \frametitle{Stringovi, liste, sekvence}
  \begin{itemize}
    \item string je specijalni oblik sekvence
    \item string operacije postoje za sve sekvence!
  \end{itemize}
\begin{minted}[linenos=false]{python}
>>> [1,2] + [3,4]
[1, 2, 3, 4]
>>> [1,2]*3
[1, 2, 1, 2, 1, 2]
>>> grades = ['A', 'B', 'C', 'D', 'F']
>>> grades[0]
'A'
>>> grades[2:4]
['C', 'D']
>>> len(grades)
5
\end{minted}
\end{frame}

\begin{frame}[fragile]
  \frametitle{Stringovi, liste, sekvence $_2$}
  \begin{itemize}
    \item string je uvek sekvenca znakova
    \item lista može da sadrži bilo kakve podatke -- \\ brojeve, stringove, ili oba
  \end{itemize}
\begin{minted}[linenos=false]{python}
myList = [1, "Spam ", 4, "U"]
\end{minted}
\end{frame}

\begin{frame}[fragile]
  \frametitle{Redni broj meseca u godini $\rightarrow$ troslovna oznaka $_4$}
  \begin{itemize}
    \item string je uvek sekvenca znakova
    \item lista može da sadrži bilo kakve podatke -- \\ brojeve, stringove, ili oba
  \end{itemize}
\begin{minted}[linenos=false]{python}
meseci = ["Jan", "Feb", "Mar", "Apr", "Maj", "Jun", 
    "Jul", "Avg", "Sep", "Okt", "Nov", "Dec"]
\end{minted}
  \begin{itemize}
    \item naredba dodele vrednosti se proteže preko dva reda
    \item Python zna da nije završena u prvom redu pa dalje čita tekst programa
  \end{itemize}
\end{frame}

\begin{frame}[fragile]
  \frametitle{Redni broj meseca u godini $\rightarrow$ troslovna oznaka $_5$}
  \begin{itemize}
    \item sada se naziv meseca dobija ovako:
  \end{itemize}
\begin{minted}[linenos=false]{python}
mesecSkr = meseci[n-1]
\end{minted}
  \begin{itemize}
    \item umesto ovako:
  \end{itemize}
\begin{minted}[linenos=false]{python}
mesecSkr = meseci[pos:pos+3]
\end{minted}
\end{frame}

\begin{frame}[fragile,shrink=10]
  \frametitle{Redni broj meseca u godini $\rightarrow$ troslovna oznaka $_6$}
\begin{minted}[linenos=false]{python}
# month2.py
# Štampa skraćeno ime meseca za dati redni broj

# ovde čuvamo nazive
meseci = ["Jan", "Feb", "Mar", "Apr", "Maj", "Jun", 
    "Jul", "Avg", "Sep", "Okt", "Nov", "Dec"]

n = eval(raw_input("Unesite mesec (1-12): "))

# ispiši rezultat
print ("Skraćenica je", meseci[n-1] + ".")
\end{minted}
  \begin{itemize}
    \item pošto je indeks lako izračunati, nije nam neophodna pomoćna promenljiva
  \end{itemize}
\end{frame}

\begin{frame}[fragile,shrink=10]
  \frametitle{Redni broj meseca u godini $\rightarrow$ naziv meseca}
  \begin{itemize}
    \item ova verzija programa se lako menja za rad sa punim nazivima
  \end{itemize}
\begin{minted}[linenos=false]{python}
meseci = ["Januar", "Februar", "Mart", "April", "Maj", "Jun", 
    "Jul", "Avgust", "Septembar", "Oktobar", "Novembar", 
    "Decembar"]
\end{minted}
\end{frame}

\begin{frame}[fragile,shrink=10]
  \frametitle{Menjanje sekvence, stringa i liste}
  \begin{itemize}
    \item sekvenca je lista koja se ne može menjati
    \item string je sekvenca $\Rightarrow$ ne može se menjati
    \item lista se može menjati
  \end{itemize}
\begin{minted}[linenos=false]{python}
>>> myList = [34, 26, 15, 10]
>>> myList[2]
15
>>> myList[2] = 0
>>> myList
[34, 26, 0, 10]
>>> myString = "Hello World"
>>> myString[2]
'l'
>>> myString[2] = "p"

Traceback (most recent call last):
  File "<pyshell#16>", line 1, in -toplevel-
    myString[2] = "p"
TypeError: object doesn't support item assignment
\end{minted}
\end{frame}

\section{Kodiranje slova}

\begin{frame}[fragile]
  \frametitle{Kodni rasporedi}
  \begin{itemize}
    \item \myblue{kodni raspored}: mapiranje slova na brojeve radi reprezentacije slova u memoriji
    \item u davna vremena, svaki proizvođač računara je imao sopstveni kodni raspored
    \item ASCII standard: 7-bitni kodni raspored (127 slova)
    \item Unicode standard: preko 100.000 različitih slova
    \item Python radi sa Unicode slovima (default od verzije 3)
  \end{itemize}
\end{frame}

\begin{frame}[fragile]
  \frametitle{Konverzija slovo $\leftrightarrow$ broj}
  \begin{itemize}
    \item funkcija \texttt{ord} vraća numerički kod za dato slovo
    \item funkcija \texttt{chr} vraća slovo za dati numerički kod
  \end{itemize}
\begin{minted}[linenos=false]{python}
>>> ord("A")
65
>>> ord("a")
97
>>> chr(97)
'a'
>>> chr(65)
'A'
\end{minted}
\end{frame}

\begin{frame}[fragile]
  \frametitle{Primer: konverzija string $\leftrightarrow$ niz brojeva}
  \begin{itemize}
    \item algoritam je jednostavan:
    \item[1] unesi string za konverziju
    \item[2] za svaki znak u stringu:
    \item[2a] \ \ \ ispiši numerički kod znaka \\ \ \\ %
    \item \texttt{for} petlja ide kroz sekvencu, a string je sekvenca: \\
      \texttt{for <ch> in <string>:}
  \end{itemize}
\end{frame}

\begin{frame}[fragile,shrink=10]
  \frametitle{Primer: konverzija string $\leftrightarrow$ niz brojeva $_2$}
\begin{minted}[linenos=false]{python}
# text2numbers.py
# Konvertuje string u niz brojeva koristeći Unicode
# kodni raspored

print("Konvertuje string u niz brojeva koristeći")
print("Unicode kodni raspored.\n")

# unesi tekst
message = raw_input("Unesite tekst za kodiranje: ")

print("\nOvo su Unicode kodovi:")

# za svaki znak u stringu ispiši njegov kod
for ch in message:
    print(ord(ch),  end=" ")

print()
\end{minted}
\end{frame}

\begin{frame}[fragile]
  \frametitle{Primer: konverzija niz brojeva $\leftrightarrow$ string}
  \begin{itemize}
    \item bilo bi lepo imati i program koji radi inverziju prethodnog
    \item algoritam za dekoder bi mogao da glasi:
    \item[1] unesi niz brojeva za dekodiranje
    \item[2] message $\leftarrow$ ""
    \item[3] za svaki broj u nizu:
    \item[3a] \ \ \ konvertuj broj u znak
    \item[3b] \ \ \ dodaj znak na kraj message
    \item[4] ispiši message
  \end{itemize}
\end{frame}

\begin{frame}[fragile]
  \frametitle{Primer: konverzija niz brojeva $\leftrightarrow$ string $_2$}
  \begin{itemize}
    \item promenljiva \texttt{message} je akumulator
    \item inicijalno postavljena na \myblue{prazan string}
    \item u svakom prolazu petlje jedan broj se konvertuje u znak
    \item i dodaje na kraj akumulatora
  \end{itemize}
\end{frame}

\begin{frame}[fragile]
  \frametitle{Primer: konverzija niz brojeva $\leftrightarrow$ string $_2$}
  \begin{itemize}
    \item kako da unesemo sa tastature niz brojeva promenljive dužine?
    \item učitamo ga kao string, pa podelimo na delove koji predstavljaju pojedinačne brojeve
    \item novi algoritam:
    \item[1] unesi niz brojeva kao string
    \item[2] message $\leftarrow$ ""
    \item[3] za svaki od podstringova:
    \item[3a] \ \ \ konvertuj niz znakova koji predstavljaju broj u broj
    \item[3b] \ \ \ konvertuj taj broj u znak
    \item[3c] \ \ \ dodaj znak na kraj message
    \item[4] ispiši message
  \end{itemize}
\end{frame}

\begin{frame}[fragile]
  \frametitle{Funkcija \texttt{split}}
  \begin{itemize}
    \item podeli string na podstringove, u odnosu na neki separator-znak
  \end{itemize}
\begin{minted}[linenos=false]{python}
>>> "Hello string methods!".split()
['Hello', 'string', 'methods!']

>>> "32,24,25,57".split(",")
['32', '24', '25', '57']
>>>
\end{minted}
\end{frame}

\begin{frame}[fragile]
  \frametitle{Funkcija \texttt{eval}}
  \begin{itemize}
    \item konverzija string $\rightarrow$ broj pomoću funkcije \texttt{eval}
  \end{itemize}
\begin{minted}[linenos=false]{python}
>>> numStr = "500"
>>> eval(numStr)
500
>>> x = eval(raw_input("Enter a number "))
Enter a number 3.14
>>> print x
3.14
>>> type (x)
<type 'float'>
\end{minted}
\end{frame}

\begin{frame}[fragile,shrink=10]
  \frametitle{Primer: konverzija niz brojeva $\leftrightarrow$ string $_2$}
\begin{minted}[linenos=false]{python}
# numbers2text.py
# Konvertuje niz Unicode brojeva u string

print("Konvertuje niz Unicode brojeva u string.\n")

# unesi poruku
inString = input("Unesite poruku kao Unicode niz: ")

# za svaki podstring odredi znak i dodaj ga u rezultat
message = ""
for numStr in inString.split(" "):
    # konvertuj podstring u broj
    codeNum = eval(numStr)
    # dodaj znak na kraj rezultata
    message = message + chr(codeNum) 

print("\nDekodirana poruka je:", message)
\end{minted}
\end{frame}

\begin{frame}[fragile,shrink=10]
  \frametitle{Primer: konverzija niz brojeva $\leftrightarrow$ string $_2$}
\begin{verbatim}
$ python text2numbers.py
Konvertuje string u niz brojeva koristeći
Unicode kodni raspored.

Unesite tekst za kodiranje: Pajton je zakon

Ovo su Unicode kodovi:
80 97 106 116 111 110 32 106 101 32 122 97 107 111 110

$ python numbers2text.py
Konvertuje niz Unicode brojeva u string.

Unesite poruku kao Unicode niz: 80 97 106 116 111 110 32 106 101 32 122 97 107 111 110

Dekodirana poruka je: Pajton je zakon
\end{verbatim}
\end{frame}

\begin{frame}[fragile]
  \frametitle{Još funkcija za stringove}
  \begin{itemize}
    \item postoji puno funkcija koje operišu nad stringovima
    \item \texttt{s.capitalize()} -- kopija \texttt{s} sa prvim znakom pretvorenim u veliko slovo
    \item \texttt{s.title()} -- kopija \texttt{s} sa prvim znakom svake reči pretvorenim u veliko slovo
    \item \texttt{s.center(width)} -- centriraj \texttt{s} u string dužine \texttt{width}
    \item \texttt{s.ljust(width)} -- kao \texttt{center}, samo je tekst poravnat sa leve strane
    \item \texttt{s.rjust(width)} -- kao \texttt{center}, samo je tekst poravnat sa desne strane
  \end{itemize}
\end{frame}

\begin{frame}[fragile]
  \frametitle{Još funkcija za stringove $_2$}
  \begin{itemize}
    \item \texttt{s.count(sub)} -- broj pojavljivanja podstringa \texttt{sub} u stringu \texttt{s}
    \item \texttt{s.find(sub)} -- indeks prvog mesta gde se podstring \texttt{sub} pojavljuje u stringu \texttt{s}
    \item \texttt{s.rfind(sub)} -- kao \texttt{find}, samo traži podstring sa desne strane
    \item \texttt{s.join(list)} -- konkatenraj stringove iz liste \texttt{list} koristeći \texttt{s} kao separator
  \end{itemize}
\end{frame}

\begin{frame}[fragile]
  \frametitle{Još funkcija za stringove $_3$}
  \begin{itemize}
    \item \texttt{s.lower()} -- konvertuje sve u mala slova
    \item \texttt{s.upper()} -- konvertuje sve u velika slova
    \item \texttt{s.lstrip()} -- ukloni razmake (whitespace) sa početka stringa
    \item \texttt{s.rstrip()} -- ukloni razmake (whitespace) sa kraja stringa
    \item \texttt{s.split(sep)} -- podeli \texttt{s} na delove oko znakova \texttt{sep}
    \item \texttt{s.replace(oldsub, newsub)} -- zameni sve podstringove \texttt{oldsub} u stringu \texttt{s} sa \texttt{newsub}
  \end{itemize}
\end{frame}

\section{Formatiranje}

\begin{frame}[fragile]
  \frametitle{Formatiranje stringova}
  \begin{itemize}
    \item često moramo da ,,ulepšamo`` string pre ispisa
    \item recimo da hoćemo da konvertujemo datum iz oblika ''15.10.2011.'' u oblik ''15. oktobar 2011.''
    \item[1] unesi datum u formatu dd.mm.gggg.
    \item[2] podeli string oko tačke u dan, mesec i godinu
    \item[3] konvertuj broj meseca u naziv
    \item[4] formiraj novi string u obliku dd. mesec gggg.
    \item[5] ispiši novi string
  \end{itemize}
\end{frame}

\begin{frame}[fragile]
  \frametitle{Formatiranje stringova $_2$}
  \begin{itemize}
    \item prva dva reda su jednostavna: \\
      \texttt{dateStr = input('Unesite datum (dd.mm.gggg.): ')} \\
      \texttt{dayStr, monthStr, yearStr, x = dateStr.split('.')} \\ \ \\ %
    \item datum se učita kao string, pa se podeli oko tačke na 4 dela
    \item prva tri dela upišemo u bitne promenljive
  \end{itemize}
\end{frame}

\begin{frame}[fragile]
  \frametitle{Formatiranje stringova $_3$}
  \begin{itemize}
    \item sledeći korak: konvertovati \texttt{monthStr} u broj
    \item možemo da koristimo funkciju \texttt{int} \\
      \texttt{int("05") = 5}
  \end{itemize}
\end{frame}

\begin{frame}[fragile]
  \frametitle{Formatiranje stringova $_4$}
  \begin{itemize}
    \item \texttt{eval} neće raditi za brojeve sa vodećim nulama\\ (vodeća nula je nekad označavala oktalni broj)
  \end{itemize}
\begin{verbatim}
>>> int("05")
5
>>> eval("05")
 Traceback (most recent call last):
File "<pyshell#9>", line 1, in <module>
eval("05")
File "<string>", line 1
05
     ^
SyntaxError: invalid token
\end{verbatim}
\end{frame}

\begin{frame}[fragile]
  \frametitle{Formatiranje stringova $_5$}
\begin{minted}[linenos=false]{python}
months = [“januar”, “februar”, ..., “decembar”]
monthStr = months[int(monthStr) – 1]
\end{minted}
  \begin{itemize}
    \item indeks u listi počinje od nula, zato oduzimamo 1 od broja meseca
    \item ostaje još konkatenacija rezultujućeg stringa
  \end{itemize}
\end{frame}

\begin{frame}[fragile]
  \frametitle{Formatiranje stringova $_6$}
\begin{minted}[linenos=false]{python}
print ("Konvertovani datum je:", dayStr+".", 
    monthStr, yearStr+".")
\end{minted}
  \begin{itemize}
    \item tačka je dodata na \texttt{dayStr} i \texttt{yearStr} konkatenacijom
  \end{itemize}
\begin{verbatim}
Unesite datum (dd.mmm.gggg.): 15.10.2011.
Konvertovani datum je: 15. oktobar 2011.
\end{verbatim}
\end{frame}

\begin{frame}[fragile]
  \frametitle{Broj $\rightarrow$ string}
  \begin{itemize}
    \item za konverziju broj $\rightarrow$ string možemo da koristimo funkciju \texttt{str}
  \end{itemize}
\begin{minted}[linenos=false]{python}
>>> str(500)
'500'
>>> value = 3.14
>>> str(value)
'3.14'
>>> print("Vednost je", str(value) + ".")
Vrednost je 3.14.
\end{minted}
\end{frame}

\begin{frame}[fragile,shrink=6]
  \frametitle{Broj $\rightarrow$ string i konkatenacija}
  \begin{itemize}
    \item ako je neka vrednost string, možemo konkatenirati tačku na njen kraj
    \item ako je ta vrednost int, šta onda?
  \end{itemize}
\begin{verbatim}
>>> value = 3.14
>>> print("Vrednost je", value + ".")
Vrednost je

Traceback (most recent call last):
  File "<pyshell#10>", line 1, in -toplevel-
    print "The value is", value + "."
TypeError: unsupported operand type(s) for +: 'float' and 'str'
\end{verbatim}
\end{frame}

\begin{frame}[fragile]
  \frametitle{Operacije za konverziju}
\begin{center}
\begin{tabular}{l|l}
\textbf{funkcija} & \textbf{značenje} \\ \hline
\texttt{float(expr)} & konvertuj \texttt{expr} u \texttt{float} vrednost \\
\texttt{int(expr)} & konvertuj \texttt{expr} u \texttt{int} vrednost \\
\texttt{str(expr)} & vrati string reprezentaciju od \texttt{expr} \\
\texttt{eval(str)} & izračunaj \texttt{str} kao Python izraz
\end{tabular}
\end{center}
\end{frame}

\begin{frame}[fragile]
  \frametitle{Formatiranje stringova}
\begin{verbatim}
Change Counter

Please enter the count of each coin type.
Quarters: 6
Dimes: 0
Nickels: 0
Pennies: 0

The total value of your change is 1.5
\end{verbatim}
  \begin{itemize}
    \item bilo bi lepše da na kraju piše \texttt{\$1.50}
  \end{itemize}
\end{frame}

\begin{frame}[fragile]
  \frametitle{Formatiranje stringova $_2$}
  \begin{itemize}
    \item možemo da preradimo ispis rezultata ovako:
  \end{itemize}
\begin{verbatim}
print("Your change is ${0:0.2f}".format(total))
\end{verbatim}
  \begin{itemize}
    \item sada ćemo dobiti nešto ovako:
  \end{itemize}
\begin{verbatim}
The total value of your change is $1.50
\end{verbatim}
\end{frame}

\begin{frame}[fragile]
  \frametitle{Funkcija \texttt{format}}
  \begin{itemize}
    \item \texttt{<šablon>.format(<vrednosti>)}
    \item vitičaste zagrade \{...\} označavaju mesta u koja se upisuju vrednosti
    \item svako mesto ima \myblue{format}
  \end{itemize}
\end{frame}

\begin{frame}[fragile]
  \frametitle{Funkcija \texttt{format} $_2$}
\begin{verbatim}
print("Your change is ${0:0.2f}".format(total))
\end{verbatim}
  \begin{itemize}
    \item šablon sadrži jedno mesto sa opisom \texttt{0:0.2f}
    \item opis ima format \texttt{<index>:<format-spec>}
    \item \texttt{index} je redni broj parametra koga treba upisati
    \item \texttt{format-spec} ima oblik: \texttt{<width>.<precision><type>}
    \item \texttt{width} je broj znakova koju vrednost treba da zauzima; \\0 znači da zauzme koliko je potrebno
    \item \texttt{precision} je broj decimala
    \item \texttt{f} znači broj u ,,fiksnom zarezu``
  \end{itemize}
\end{frame}

\begin{frame}[fragile,shrink=25]
  \frametitle{Funkcija \texttt{format} $_3$}
\begin{verbatim}
>>> "Hello {0} {1}, you may have won ${2}" .format("Mr.", "Smith", 10000)
'Hello Mr. Smith, you may have won $10000'

>>> 'This int, {0:5}, was placed in a field of width 5'.format(7)
'This int,     7, was placed in a field of width 5'

>>> 'This int, {0:10}, was placed in a field of witdh 10'.format(10)
'This int,         10, was placed in a field of witdh 10'

>>> 'This float, {0:10.5}, has width 10 and precision 5.'.format(3.1415926)
'This float,    3.1416, has width 10 and precision 5.'

>>> 'This float, {0:10.5f},  is fixed at 5 decimal places.'.format(3.1415926)
'This float,   3.14159, has width 0 and precision 5.'
\end{verbatim}
\end{frame}

\begin{frame}[fragile]
  \frametitle{Funkcija \texttt{format} $_4$}
  \begin{itemize}
    \item ako je \texttt{width} veči nego što je potrebno
    \begin{itemize}
      \item podrazumevano poravnanje:
      \item numerički podaci se poravnavaju po desnoj strani
      \item stringovi se poravnavaju po levoj strani
    \end{itemize}
    \item poravnanje se može eksplicitno navesti pre \texttt{width}
  \end{itemize}
\begin{verbatim}
>>> "levo poravnato: {0:<5}.format("Hi!")
'levo poravnato: Hi!  '
>>> "desno poravnato: {0:>5}.format("Hi!")
'desno poravnato:   Hi!'
>>> "centrirano: {0:^5}".format("Hi!")
'centrirano:  Hi! '
\end{verbatim}
\end{frame}

\begin{frame}[fragile]
  \frametitle{Novčani iznosi}
  \begin{itemize}
    \item brojevi u pokretnom zarezu su nezgodni za čuvanje novčanih iznosa
    \item možemo čuvati novac u parama/centima kao \texttt{int} i konvertovati u dinare/evre i pare/cente prilikom ispisa
    \item ako je \texttt{total} vrednost u evro centima:
  \end{itemize}
\begin{verbatim}
euros = total // 100
cents = total % 100
\end{verbatim}
  \begin{itemize}
    \item centi se ispisuju sa širinom \texttt{0>2} da bismo dobili desno poravnat broj sa vodećim nulama
    \item tako se 5 centi ispisuje kao \texttt{05}
  \end{itemize}
\end{frame}

\begin{frame}[fragile]
  \frametitle{Novčani iznosi $_2$}
\begin{minted}[linenos=false]{python}
# change2.py

print ("Change Counter\n")
print ("Please enter the count of each coin type.")
quarters = eval(raw_input("Quarters: "))
dimes = eval(raw_input("Dimes: "))
nickels = eval(raw_input("Nickels: "))
pennies = eval(raw_input("Pennies: "))
total = quarters * 25 + dimes * 10 + nickels * 5 + pennies 

print ("The total value of your change is ${0}.{1:0>2}"
    .format(total//100, total%100))
\end{minted}
\end{frame}

\begin{frame}[fragile,shrink=20]
  \frametitle{Novčani iznosi $_3$}
\begin{center}
\begin{tabular}{p{6cm}|p{6cm}}
\begin{verbatim}
Change Counter

Please enter the count of
each coin type.
Quarters: 0
Dimes: 0
Nickels: 0
Pennies: 1

The total value of your change
is $0.01
\end{verbatim}
&
\begin{verbatim}
Change Counter

Please enter the count of
each coin type.
Quarters: 12
Dimes: 1
Nickels: 0
Pennies: 4

The total value of your change
is $3.14
\end{verbatim}
\end{tabular}
\end{center}
\end{frame}

\section{Fajlovi}

\begin{frame}[fragile]
  \frametitle{Fajlovi}
  \begin{itemize}
    \item \myblue{fajl} (datoteka) je sekvenca podataka koji se čuva na sekundarnoj memoriji (disku)
    \item fajlovi mogu da sadrže bilo koji tip podataka
    \item najlakše je raditi sa tekstom
    \item tekstualni fajl tipično sadrži više redova teksta
    \item Python koristi \myblue{line feed} znak (\texttt{\textbackslash n}, str(10)) da označi kraj reda
  \end{itemize}
\end{frame}

\begin{frame}[fragile]
  \frametitle{Višelinijski stringovi}
\begin{verbatim}
Hello
World

Goodbye 32
\end{verbatim}
  \begin{itemize}
    \item kada se ovaj string snimi u fajl, on izgleda ovako:
  \end{itemize}
\begin{verbatim}
Hello\nWorld\n\nGoodbye 32\n
\end{verbatim}
\end{frame}

\begin{frame}[fragile]
  \frametitle{Višelinijski stringovi}
  \begin{itemize}
    \item ovo radi na isti način kao ugrađivanje \texttt{\textbackslash n} u \texttt{print} naredbe
    \item ovi znakovi utiču samo na ispis; ne utiču na izračunavanje izraza
  \end{itemize}
\end{frame}

\begin{frame}[fragile]
  \frametitle{Obrađivanje fajlova}
  \begin{itemize}
    \item proces \myblue{otvaranja} fajla uključuje povezivanje fajla na disku sa objektom u memoriji
    \item nakon otvaranja možemo menjati fajl putem tog objekta
    \begin{itemize}
      \item čitanje iz fajla
      \item pisanje u fajl
    \end{itemize}
    \item nakon obavljenog posla, fajl je potrebno \myblue{zatvoriti}
  \end{itemize}
\end{frame}

\begin{frame}[fragile]
  \frametitle{Zatvaranje fajla}
  \begin{itemize}
    \item zatvaranje fajla će pokrenuti sve operacije nad fajlom koje do tada nisu izvršene
    \begin{itemize}
      \item snimanje podataka
      \item pražnjenje bafera, itd.
    \end{itemize}
    \item ako zaboravimo da zatvorimo fajl, moguće je da će doći do gubitka nekih podataka!
  \end{itemize}
\end{frame}

\begin{frame}[fragile]
  \frametitle{Čitanje fajla}
  \begin{itemize}
    \item učitavanje fajla u tekst-editor
    \begin{itemize}
      \item otvaranje fajla
      \item učitavanje sadržaja fajla u memoriju
      \item zatvaranje fajla
      \item izmene nad sadržajem se obavljaju u memoriji, fajl se ne menja!
    \end{itemize}
  \end{itemize}
\end{frame}

\begin{frame}[fragile]
  \frametitle{Snimanje fajla}
  \begin{itemize}
    \item snimanje fajla iz memorije na disk
    \begin{itemize}
      \item originalni fajl je otvara u režimu koji omogućuje upis podataka -- to će obrisati stari sadržaj!
      \item operacije za upis podataka će iskopirati sadržaj iz memorije na disk
      \item zatvaranje fajla (obavezno!)
    \end{itemize}
  \end{itemize}
\end{frame}

\begin{frame}[fragile]
  \frametitle{Tekstualni fajlovi i Python}
  \begin{itemize}
    \item otvaranje fajla (kreiranje objekta u memoriji koji predstavlja fajl) \\
      \texttt{<filevar> = open(<name>, <mode>)}
    \item \texttt{name} je string koji predstavlja ime fajla na disku
    \item \texttt{mode} je \texttt{"r"} ili \texttt{"w"} zavisno od toga da li čitamo ili pišemo u fajl
  \end{itemize}
\begin{minted}[linenos=false]{python}
infile = open("numbers.dat", "r")
outfile = open("numbers2.dat", "w")
\end{minted}
\end{frame}

\begin{frame}[fragile]
  \frametitle{Operacije nad fajlom}
  \begin{itemize}
    \item \texttt{<file>.read()} -- vraća preostali sadržaj fajla kao jedan (veliki!) višelinijski string
      \texttt{<filevar> = open(<name>, <mode>)}
    \item \texttt{<file>.readline()} -- vraća narednu liniju teksta iz fajla uključujući i \texttt{\textbackslash n}
    \item \texttt{<file>.readlines()} -- vraća listu preostalih linija teksta iz fajla; elementi liste su redovi teksta koji se završavaju sa \texttt{\textbackslash n}
  \end{itemize}
\end{frame}

\begin{frame}[fragile]
  \frametitle{Primer operacija nad fajlom}
\begin{minted}[linenos=false]{python}
# printfile.py

fname = input("Enter filename: ")
infile = open(fname,'r')
data = infile.read()
print(data)
infile.close()
\end{minted}
  \begin{itemize}
    \item[1] korisnik unosi ime fajla
    \item[2] fajl se otvara za čitanje
    \item[3] ceo fajl se učitava i smešta u string \texttt{data}
    \item[4] string \texttt{data} se ispisuje
  \end{itemize}
\end{frame}

\begin{frame}[fragile]
  \frametitle{Čitanje celog fajla}
  \begin{itemize}
    \item \texttt{readline} se može upotrebiti za čitanje sledećeg reda iz fajla, uključujući i \texttt{\textbackslash n}
  \end{itemize}
\begin{minted}[linenos=false]{python}
infile = open(someFile, "r")
for i in range(5):
    line = infile.readline()
    print(line[:-1])
infile.close()
\end{minted}
  \begin{itemize}
    \item čita prvih 5 redova iz fajla
    \item uklanjanje \texttt{\textbackslash n} se obavlja pomoću \texttt{split}
  \end{itemize}
\end{frame}

\begin{frame}[fragile]
  \frametitle{Čitanje celog fajla $_2$}
  \begin{itemize}
    \item drugi način je da pročitamo sve redove pomoću \texttt{readlines}
    \item i da u petlji prođemo kroz listu stringova
  \end{itemize}
\begin{minted}[linenos=false]{python}
infile = open(someFile, "r")
for line in infile.readlines():
    # uradi nešto sa tekućim redom
infile.close()
\end{minted}
\end{frame}


\begin{frame}[fragile]
  \frametitle{Čitanje celog fajla $_3$}
  \begin{itemize}
    \item Python tretira fajl-objekat kao sekvencu redova!
  \end{itemize}
\begin{minted}[linenos=false]{python}
infile = open(someFile, "r")
for line in infile:
    # uradi nešto sa tekućim redom
infile.close()
\end{minted}
\end{frame}

\begin{frame}[fragile]
  \frametitle{Pisanje u fajl}
  \begin{itemize}
    \item otvaranje fajla za upis priprema fajl za prijem podataka
    \item ako se otvori postojeći fajl, njegov sadržaj će biti obrisan
    \item ako fajl prethodno ne postoji, biće kreiran novi fajl
  \end{itemize}
\begin{minted}[linenos=false]{python}
outfile = open(someFile, "w")
print(sadržaj, file=outfile)
\end{minted}
\end{frame}

\begin{frame}[fragile]
  \frametitle{Batch režim rada}
  \begin{itemize}
    \item \myblue{batch} režim rada: nema neposrednog unosa/ispisa podataka
    \item ulaz i izlaz se odvija preko fajlova
    \item primer: kreiranje korisničkih imena (username) na osnovu imena i prezimena
  \end{itemize}
\end{frame}

\begin{frame}[fragile]
  \frametitle{Batch režim rada $_2$}
\begin{minted}[linenos=false]{python}
infileName = input("Input file? ")
outfileName = input("Output file? ")

infile = open(infileName, 'r')
outfile = open(outfileName, 'w')

for line in infile:
    first, last = line.split()
    uname = (first[0]+last[:7]).lower()
    print(uname, file=outfile)

infile.close()
outfile.close()

print("Written to", outfileName)
\end{minted}
\end{frame}

\begin{frame}[fragile]
  \frametitle{Batch režim rada $_3$}
  \begin{itemize}
    \item programi često imaju otvoreno više fajlova
    \item funkcija \texttt{lower} je zgodna za konverziju imena fajlova u mala slova
  \end{itemize}
\end{frame}

\end{document}
